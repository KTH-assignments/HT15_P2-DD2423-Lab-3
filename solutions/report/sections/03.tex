% --------------------------------- Question 7 ---------------------------------
\subsection{Question 7}

The ideal parameter setting for each of these images depend on various
factors, along with the degree of segmentation that is desired.

For example, parameter \texttt{min\_area} depends on the size of the
image, the size of the objects that are to be segmented and the complexity
of each object's structure. Figures \ref{fig:03_Q7_orange_ma_500} -
\ref{fig:03_Q7_orange_ma_10} illustrate the effect that the variation of the
value of \texttt{min\_area} has on image \texttt{orange}. Notice that
when \texttt{min\_area} $\geq 100$ the center of the left half of the orange
cannot be found and is taken over by its surroundings.

\noindent\makebox[\textwidth][c]{%
\begin{minipage}{\linewidth}
  \begin{minipage}{0.45\linewidth}
    \begin{figure}[H]
      \includegraphics[scale=0.8]{./images/03/orange/ma/normcuts1_ma_500.png}
      \caption{Image \texttt{orange} segmented with the default settings and
        \texttt{min\_area = 500}}
      \label{fig:03_Q7_orange_ma_500}
    \end{figure}
  \end{minipage}
  \hfill
  \begin{minipage}{0.45\linewidth}
    \begin{figure}[H]
      \includegraphics[scale=0.8]{./images/03/orange/ma/normcuts1_ma_200.png}
      \caption{Image \texttt{orange} segmented with the default settings and
        \texttt{min\_area = 200}}
      \label{fig:03_Q7_orange_ma_200}
    \end{figure}
  \end{minipage}
\end{minipage}
}

\noindent\makebox[\textwidth][c]{%
\begin{minipage}{\linewidth}
  \begin{minipage}{0.45\linewidth}
    \begin{figure}[H]
      \includegraphics[scale=0.8]{./images/03/orange/ma/normcuts1_ma_100.png}
      \caption{Image \texttt{orange} segmented with the default settings and
        \texttt{min\_area = 100}}
      \label{fig:03_Q7_orange_ma_100}
    \end{figure}
  \end{minipage}
  \hfill
  \begin{minipage}{0.45\linewidth}
    \begin{figure}[H]
      \includegraphics[scale=0.8]{./images/03/orange/ma/normcuts1_ma_10.png}
      \caption{Image \texttt{orange} segmented with the default settings and
        \texttt{min\_area = 10}}
      \label{fig:03_Q7_orange_ma_10}
    \end{figure}
  \end{minipage}
\end{minipage}
}
\\

Parameter \texttt{ncut\_thresh} depends on the degree of derisable segmentation,
the spatial complexity and the diversity in colour of each image. Figures
\ref{fig:03_Q7_tiger1_nt_001} - \ref{fig:03_Q7_tiger1_nt_05} illustrate the
effect that the increase in the maximum allowed value for a cut to made has
on image \texttt{tiger1}. This image has significantly more colour diversity
than image \texttt{orange}.


\noindent\makebox[\textwidth][c]{%
\begin{minipage}{\linewidth}
  \begin{minipage}{0.45\linewidth}
    \begin{figure}[H]
      \includegraphics[scale=0.8]{./images/03/tiger1/nt/normcuts1_nt_0.01.png}
      \caption{Image \texttt{tiger1} segmented with the default settings and
        \texttt{ncuts\_thresh = 0.01}}
      \label{fig:03_Q7_tiger1_nt_001}
    \end{figure}
  \end{minipage}
  \hfill
  \begin{minipage}{0.45\linewidth}
    \begin{figure}[H]
      \includegraphics[scale=0.8]{./images/03/tiger1/nt/normcuts1_nt_0.02.png}
      \caption{Image \texttt{tiger1} segmented with the default settings and
        \texttt{ncuts\_thresh = 0.02}}
      \label{fig:03_Q7_tiger1_nt_002}
    \end{figure}
  \end{minipage}
\end{minipage}
}

\noindent\makebox[\textwidth][c]{%
\begin{minipage}{\linewidth}
  \begin{minipage}{0.45\linewidth}
    \begin{figure}[H]
      \includegraphics[scale=0.8]{./images/03/tiger1/nt/normcuts1_nt_0.05.png}
      \caption{Image \texttt{tiger1} segmented with the default settings and
        \texttt{ncuts\_thresh = 0.05}}
      \label{fig:03_Q7_tiger1_nt_005}
    \end{figure}
  \end{minipage}
  \hfill
  \begin{minipage}{0.45\linewidth}
    \begin{figure}[H]
      \includegraphics[scale=0.8]{./images/03/tiger1/nt/normcuts1_nt_0.1.png}
      \caption{Image \texttt{tiger1} segmented with the default settings and
        \texttt{ncuts\_thresh = 0.1}}
      \label{fig:03_Q7_tiger1_nt_01}
    \end{figure}
  \end{minipage}
\end{minipage}
}

\noindent\makebox[\textwidth][c]{%
\begin{minipage}{\linewidth}
  \begin{minipage}{0.45\linewidth}
    \begin{figure}[H]
      \includegraphics[scale=0.8]{./images/03/tiger1/nt/normcuts1_nt_0.2.png}
      \caption{Image \texttt{tiger1} segmented with the default settings and
        \texttt{ncuts\_thresh = 0.2}}
      \label{fig:03_Q7_tiger1_nt_02}
    \end{figure}
  \end{minipage}
  \hfill
  \begin{minipage}{0.45\linewidth}
    \begin{figure}[H]
      \includegraphics[scale=0.8]{./images/03/tiger1/nt/normcuts1_nt_0.5.png}
      \caption{Image \texttt{tiger1} segmented with the default settings and
        \texttt{ncuts\_thresh = 0.5}}
      \label{fig:03_Q7_tiger1_nt_05}
    \end{figure}
  \end{minipage}
\end{minipage}
}\\


Figures \ref{fig:03_Q7_tiger3_md_1} - \ref{fig:03_Q7_tiger3_md_16} illustrate
the effect that the increase in the recursion depth has on image \texttt{tiger3}.
Parameter \texttt{max\_depth}.


\noindent\makebox[\textwidth][c]{%
\begin{minipage}{\linewidth}
  \begin{minipage}{0.45\linewidth}
    \begin{figure}[H]
      \includegraphics[scale=0.8]{./images/03/tiger3/md/normcuts1_md_1.png}
      \caption{Image \texttt{tiger3} segmented with the default settings and
        \texttt{max\_depth = 1}}
      \label{fig:03_Q7_tiger3_md_1}
    \end{figure}
  \end{minipage}
  \hfill
  \begin{minipage}{0.45\linewidth}
    \begin{figure}[H]
    \includegraphics[scale=0.8]{./images/03/tiger3/md/normcuts1_md_2.png}
      \caption{Image \texttt{tiger3} segmented with the default settings and
        \texttt{max\_depth = 2}}
      \label{fig:03_Q7_tiger3_md_2}
    \end{figure}
  \end{minipage}
\end{minipage}
}

\noindent\makebox[\textwidth][c]{%
\begin{minipage}{\linewidth}
  \begin{minipage}{0.45\linewidth}
    \begin{figure}[H]
      \includegraphics[scale=0.8]{./images/03/tiger3/md/normcuts1_md_4.png}
      \caption{Image \texttt{tiger3} segmented with the default settings and
        \texttt{max\_depth = 4}}
      \label{fig:03_Q7_tiger3_md_4}
    \end{figure}
  \end{minipage}
  \hfill
  \begin{minipage}{0.45\linewidth}
    \begin{figure}[H]
    \includegraphics[scale=0.8]{./images/03/tiger3/md/normcuts1_md_8.png}
      \caption{Image \texttt{tiger3} segmented with the default settings and
        \texttt{max\_depth = 8}}
      \label{fig:03_Q7_tiger3_md_8}
    \end{figure}
  \end{minipage}
\end{minipage}
}

\noindent\makebox[\textwidth][c]{%
\begin{minipage}{\linewidth}
  \begin{minipage}{0.45\linewidth}
    \begin{figure}[H]
      \includegraphics[scale=0.8]{./images/03/tiger3/md/normcuts1_md_10.png}
      \caption{Image \texttt{tiger3} segmented with the default settings and
        \texttt{max\_depth = 10}}
      \label{fig:03_Q7_tiger3_md_10}
    \end{figure}
  \end{minipage}
  \hfill
  \begin{minipage}{0.45\linewidth}
    \begin{figure}[H]
    \includegraphics[scale=0.8]{./images/03/tiger3/md/normcuts1_md_16.png}
      \caption{Image \texttt{tiger3} segmented with the default settings and
        \texttt{max\_depth = 16}}
      \label{fig:03_Q7_tiger3_md_16}
    \end{figure}
  \end{minipage}
\end{minipage}
}\\


The combination of the three parameters
\texttt{(min\_area, ncuts\_thresh, max\_depth)} will be different for the $4$
images to the extent of the different attributes of each image. For instance
figures \ref{fig:03_Q7_orange_1_10.02.10} - \ref{fig:03_Q7_orange_2_10.05.16}
show that not only \texttt{min\_area} has to be small $(\sim 10)$ in order
for the center of the left-half of the orange to be depicted, but also higher
values are needed for \texttt{ncuts\_thresh} and \texttt{max\_depth} in order
for a more accurate segmentation to take place. The latter can be also said
for images \texttt{tiger\{1,2,3\}}, since their colour and spatial diversity
is higher than those of image \texttt{orange}, and more cuts need to be made
in order to depict the little details in the depicted animal's skin.


Figures \ref{fig:03_Q7_orange_1_10.02.10} - \ref{fig:03_Q7_tiger3_2_10.05.10}
illustrate the most reasonably well segmented results
of applying the Normal Cut segmentation method to
images \texttt{orange}, \texttt{tiger\{1,2,3\}}.



\subsubsection{Best results - image \texttt{orange}}

\noindent\makebox[\textwidth][c]{%
\begin{minipage}{\linewidth}
  \begin{minipage}{0.45\linewidth}
    \begin{figure}[H]
      \includegraphics[scale=0.8]{./images/03/orange/best/normcuts1_ma_10_nt_0.2_md_10.png}
      \caption{Image \texttt{orange} segmented with
        \texttt{(min\_area, ncut\_thresh, max\_depth) $\equiv$ (10, 0.2, 10)}.}
      \label{fig:03_Q7_orange_1_10.02.10}
    \end{figure}
  \end{minipage}
  \hfill
  \begin{minipage}{0.45\linewidth}
    \begin{figure}[H]
      \includegraphics[scale=0.8]{./images/03/orange/best/normcuts2_ma_10_nt_0.2_md_10.png}
      \caption{Image \texttt{orange} and the bounds of its segments.
        \texttt{(min\_area, ncut\_thresh, max\_depth) $\equiv$ (10, 0.2, 10)}.}
      \label{fig:03_Q7_orange_2_10.02.10}
    \end{figure}
  \end{minipage}
\end{minipage}
}

\noindent\makebox[\textwidth][c]{%
\begin{minipage}{\linewidth}
  \begin{minipage}{0.45\linewidth}
    \begin{figure}[H]
      \includegraphics[scale=0.8]{./images/03/orange/best/normcuts1_ma_10_nt_0.5_md_16.png}
      \caption{Image \texttt{orange} segmented with
        \texttt{(min\_area, ncut\_thresh, max\_depth) $\equiv$ (10, 0.5, 16)}.}
      \label{fig:03_Q7_orange_1_10.05.16}
    \end{figure}
  \end{minipage}
  \hfill
  \begin{minipage}{0.45\linewidth}
    \begin{figure}[H]
      \includegraphics[scale=0.8]{./images/03/orange/best/normcuts2_ma_10_nt_0.5_md_16.png}
      \caption{Image \texttt{orange} and the bounds of its segments.
        \texttt{(min\_area, ncut\_thresh, max\_depth) $\equiv$ (10, 0.5, 16)}.}
      \label{fig:03_Q7_orange_2_10.05.16}
    \end{figure}
  \end{minipage}
\end{minipage}
}


\subsubsection{Image \texttt{tiger1}}

\noindent\makebox[\textwidth][c]{%
\begin{minipage}{\linewidth}
  \begin{minipage}{0.45\linewidth}
    \begin{figure}[H]
      \includegraphics[scale=0.8]{./images/03/tiger1/best/normcuts1_ma_10_nt_0.1_md_8.png}
      \caption{Image \texttt{tiger1} segmented with
        \texttt{(min\_area, ncut\_thresh, max\_depth) $\equiv$ (10, 0.1, 8)}.}
      \label{fig:03_Q7_tiger1_1_10.01.8}
    \end{figure}
  \end{minipage}
  \hfill
  \begin{minipage}{0.45\linewidth}
    \begin{figure}[H]
      \includegraphics[scale=0.8]{./images/03/tiger1/best/normcuts2_ma_10_nt_0.1_md_8.png}
      \caption{Image \texttt{tiger1} and the bounds of its segments.
        \texttt{(min\_area, ncut\_thresh, max\_depth) $\equiv$ (10, 0.1, 8)}.}
      \label{fig:03_Q7_tiger1_2_10.01.8}
    \end{figure}
  \end{minipage}
\end{minipage}
}

\noindent\makebox[\textwidth][c]{%
\begin{minipage}{\linewidth}
  \begin{minipage}{0.45\linewidth}
    \begin{figure}[H]
      \includegraphics[scale=0.8]{./images/03/tiger1/best/normcuts1_ma_10_nt_0.2_md_16.png}
      \caption{Image \texttt{tiger1} segmented with
        \texttt{(min\_area, ncut\_thresh, max\_depth) $\equiv$ (10, 0.2, 16)}.}
      \label{fig:03_Q7_tiger1_1_10.02.16}
    \end{figure}
  \end{minipage}
  \hfill
  \begin{minipage}{0.45\linewidth}
    \begin{figure}[H]
      \includegraphics[scale=0.8]{./images/03/tiger1/best/normcuts2_ma_10_nt_0.2_md_16.png}
      \caption{Image \texttt{tiger1} and the bounds of its segments.
        \texttt{(min\_area, ncut\_thresh, max\_depth) $\equiv$ (10, 0.2, 16)}.}
      \label{fig:03_Q7_tiger1_2_10.02.16}
    \end{figure}
  \end{minipage}
\end{minipage}
}


\subsubsection{Image \texttt{tiger2}}

\noindent\makebox[\textwidth][c]{%
\begin{minipage}{\linewidth}
  \begin{minipage}{0.45\linewidth}
    \begin{figure}[H]
      \includegraphics[scale=0.8]{./images/03/tiger2/best/normcuts1_ma_10_nt_0.5_md_10.png}
      \caption{Image \texttt{tiger2} segmented with
        \texttt{(min\_area, ncut\_thresh, max\_depth) $\equiv$ (10, 0.5, 10)}.}
      \label{fig:03_Q7_tiger2_1_10.05.10}
    \end{figure}
  \end{minipage}
  \hfill
  \begin{minipage}{0.45\linewidth}
    \begin{figure}[H]
      \includegraphics[scale=0.8]{./images/03/tiger2/best/normcuts2_ma_10_nt_0.5_md_10.png}
      \caption{Image \texttt{tiger2} and the bounds of its segments.
        \texttt{(min\_area, ncut\_thresh, max\_depth) $\equiv$ (10, 0.5, 10)}.}
      \label{fig:03_Q7_tiger2_2_10.05.10}
    \end{figure}
  \end{minipage}
\end{minipage}
}

\noindent\makebox[\textwidth][c]{%
\begin{minipage}{\linewidth}
  \begin{minipage}{0.45\linewidth}
    \begin{figure}[H]
      \includegraphics[scale=0.8]{./images/03/tiger2/best/normcuts1_ma_10_nt_0.5_md_16.png}
      \caption{Image \texttt{tiger2} segmented with
        \texttt{(min\_area, ncut\_thresh, max\_depth) $\equiv$ (10, 0.5, 16)}.}
      \label{fig:03_Q7_tiger2_1_10.05.16}
    \end{figure}
  \end{minipage}
  \hfill
  \begin{minipage}{0.45\linewidth}
    \begin{figure}[H]
      \includegraphics[scale=0.8]{./images/03/tiger2/best/normcuts2_ma_10_nt_0.5_md_16.png}
      \caption{Image \texttt{tiger2} and the bounds of its segments.
        \texttt{(min\_area, ncut\_thresh, max\_depth) $\equiv$ (10, 0.5, 16)}.}
      \label{fig:03_Q7_tiger2_2_10.05.16}
    \end{figure}
  \end{minipage}
\end{minipage}
}



\subsubsection{Image \texttt{tiger3}}

\noindent\makebox[\textwidth][c]{%
\begin{minipage}{\linewidth}
  \begin{minipage}{0.45\linewidth}
    \begin{figure}[H]
      \includegraphics[scale=0.8]{./images/03/tiger3/best/normcuts1_ma_100_nt_0.1_md_8.png}
      \caption{Image \texttt{tiger3} segmented with
        \texttt{(min\_area, ncut\_thresh, max\_depth) $\equiv$ (100, 0.1, 8)}.}
      \label{fig:03_Q7_tiger3_1_100.01.8}
    \end{figure}
  \end{minipage}
  \hfill
  \begin{minipage}{0.45\linewidth}
    \begin{figure}[H]
      \includegraphics[scale=0.8]{./images/03/tiger3/best/normcuts2_ma_100_nt_0.1_md_8.png}
      \caption{Image \texttt{tiger3} and the bounds of its segments.
        \texttt{(min\_area, ncut\_thresh, max\_depth) $\equiv$ (100, 0.1, 8)}.}
      \label{fig:03_Q7_tiger3_2_100.01.8}
    \end{figure}
  \end{minipage}
\end{minipage}
}

\noindent\makebox[\textwidth][c]{%
\begin{minipage}{\linewidth}
  \begin{minipage}{0.45\linewidth}
    \begin{figure}[H]
      \includegraphics[scale=0.8]{./images/03/tiger3/best/normcuts1_ma_10_nt_0.5_md_10.png}
      \caption{Image \texttt{tiger3} segmented with
        \texttt{(min\_area, ncut\_thresh, max\_depth) $\equiv$ (10, 0.5, 10)}.}
      \label{fig:03_Q7_tiger3_1_10.05.10}
    \end{figure}
  \end{minipage}
  \hfill
  \begin{minipage}{0.45\linewidth}
    \begin{figure}[H]
      \includegraphics[scale=0.8]{./images/03/tiger3/best/normcuts2_ma_10_nt_0.5_md_10.png}
      \caption{Image \texttt{tiger3} and the bounds of its segments.
        \texttt{(min\_area, ncut\_thresh, max\_depth) $\equiv$ (10, 0.5, 10)}.}
      \label{fig:03_Q7_tiger3_2_10.05.10}
    \end{figure}
  \end{minipage}
\end{minipage}
}



% --------------------------------- Question 8 ---------------------------------
\subsection{Question 8}
The two parameters that were able to reduce the number of cuts while at the
same time keeping a reasonable segmentation accuracy were both the
\texttt{ncuts\_thresh} and \texttt{max\_depth} parameters as it can be seen
in figures \ref{fig:03_Q7_tiger1_nt_001} - \ref{fig:03_Q7_tiger3_md_16}.
Furthermore, parameter \texttt{min\_area} can also affect the subdivision
when increased, although not in the same capacity as the former two parameters.


% --------------------------------- Question 9 ---------------------------------
\subsection{Question 9}
% --------------------------------- Question 10 ---------------------------------
\subsection{Question 10}
